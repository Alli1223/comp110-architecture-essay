\documentclass{scrartcl}

\usepackage[hidelinks]{hyperref}
\usepackage[none]{hyphenat}

\title{Essay Proposal}
\subtitle{COMP110 - Computer Architecture Essay}

\author{AR185160}

\begin{document}

\maketitle

\section*{Topic}

My essay will be on procedural level generation for a 2D platform game.

\section*{Paper 1}

\begin{description}
\item[Title:] Experience-Driven Procedural Content Generation
\item[Citation:] \cite{yannakakis2011}
\item[Abstract:] Procedural content generation (PCG) is an increasingly important area of technology within modern human-computer interaction (HCI) design. Personalization of user experience via affective and cognitive modeling, coupled with real-time adjustment of the content according to user needs and preferences are important steps toward effective and meaningful PCG. Games, Web 2.0, interface, and software design are among the most popular applications of automated content generation. The paper provides a taxonomy of PCG algorithms and introduces a framework for PCG driven by computational models of user experience. This approach, which we call Experience-Driven Procedural Content Generation (EDPCG), is generic and applicable to various subareas of HCI. We employ games as an example indicative of rich HCI and complex affect elicitation, and demonstrate the approach's effectiveness via dissimilar successful studies.

\item[Web link:] \url{http://dl.acm.org/citation.cfm?id=2052020&CFID=557770545&CFTOKEN=87119653}
\item[Full text link:] \url{http://julian.togelius.com/Yannakakis2011Experiencedriven.pdf}
\item[Comments:] I found this article as a reference from the article ``Design metaphors for procedural content generation in games'' and it is important because it gives a good general overview of the different types of procedural generation, and goes into depth about what PCG is, as well as how users interact with PCG.
\end{description}

\section*{Paper 2}
\begin{description}
\item[Title:] Launchpad: A Rhythm-Based Level Generator
for 2-D Platformers
\item[Citation:] \cite{smith2009}
\item[Abstract:] Launchpad is an autonomous level generator that is
based on a formal model of 2-D platformer level design. Levels are
built out of small segments called “rhythm groups,” which are generated
using a two-tiered, grammar-based approach. These segments
are pieced together into complete levels that are then rated
according to a set of design heuristics. Generation can be controlled
using a set of parameters that influence the level pacing and geometry.
The approach minimizes the amount of content that must be
manually authored: instead of piecing together large segments of a
level, Launchpad uses base components that are commonly found
in a number of 2-D platformers. Launchpad produces an impressive
variety of levels which are all guaranteed to be playable.

\item[Web link:] \url{http://ieeexplore.ieee.org/xpl/articleDetails.jsp?arnumber=5648340&newsearch=true&queryText=Rhythm-based%20level%20generation}
\item[Full text link:] \url{https://users.soe.ucsc.edu/~ejw/papers/Smith-Launchpad-TCIAIG-2011.pdf}
\item[Comments:] I found this article by searching the IEEEexplore website database, and it is relevant because it goes into detail about their own research into a way to generate PCG called Launchpad, which is specifically about a 2D platformer.
\end{description}

\section*{Paper 3}
\begin{description}
\item[Title:] Procedural level generation using occupancy-regulated extension
\item[Citation:] \cite{mawhorter2010}
\item[Abstract:] Existing approaches to procedural level generation in 2D platformer games are, with some notable exceptions, procedures designed to do the work of a human game designer. They offer the usual benefits and disadvantages of AI applied to a cognitive task: they can work much faster than a human level designer, and are in some cases able to explore the design space automatically to find levels with desirable qualities. But they aren't able to capture the human creativity that produces the most interesting level designs, and they are usually very specific to their particular domain. This paper introduces occupancy-regulated extension (ORE), a general geometry assembly algorithm that supports human-design-based level authoring at arbitrary scales.

\item[Web link:]\url{http://ieeexplore.ieee.org/xpl/articleDetails.jsp?arnumber=5593333&newsearch=true&queryText=2d%20platformer%20procedural%20generation}
\item[Full text link:] \url{http://citeseerx.ist.psu.edu/viewdoc/download?doi=10.1.1.379.849&rep=rep1&type=pdf}
\item[Comments:] Again I found this article from the IEEExplore database list of PCG articles, it is relevant to the essay because it is again another unique way to make PCG for a 2D platformer.
\end{description}

\section*{Paper 4}
\begin{description}
\item[Title:] Procedural Content Generation Using Patterns as Objectives
\item[Citation:] \cite{dahlskog2014}
\item[Abstract:] In this paper we present a search-based approach for procedural generation of game levels that represents levels as sequences of micro-patterns and searched for meso-patterns. The micro-patterns are “slices” of original human-designed levels from an existing game, whereas the meso-patters are abstractions of common design patterns seen in the same levels. This method generates levels that are similar in style to the levels from which the original patterns were extracted, while still allowing for considerable variation in the geometry of the generated levels. The evolutionary method for generating the levels was tested extensively to investigate the distribution of micro-patterns used and meso-patterns found.

\item[Web link:] \url{http://link.springer.com/chapter/10.1007%2F978-3-662-45523-4_27}
\item[Full text link:] \url{http://julian.togelius.com/Dahlskog2014Procedural.pdf}
\item[Comments:] I found this article when looking for another article's pdf file that had a similar name, and I believe this article is relevant because it presents its own unique way to procedurally generate a 2D platformer that I will 
\end{description}

\section*{Paper 5}
\begin{description}
\item[Title:] Procedural Content Generation for Games: A Survey
\item[Citation:] \cite{hendrikx2013}
\item[Abstract:] Hundreds of millions of people play computer games every day. For them, game content—from 3D objects to abstract puzzles—plays a major entertainment role. Manual labor has so far ensured that the quality and quantity of game content matched the demands of the playing community, but is facing new scalability challenges due to the exponential growth over the last decade of both the gamer population and the production costs. Procedural Content Generation for Games (PCG-G) may address these challenges by automating, or aiding in, game content generation. PCG-G is difficult, since the generator has to create the content, satisfy constraints imposed by the artist, and return interesting instances for gamers. Despite a large body of research focusing on PCG-G, particularly over the past decade, ours is the first comprehensive survey of the field of PCG-G. We first introduce a comprehensive, six-layered taxonomy of game content: bits, space, systems, scenarios, design, and derived. Second, we survey the methods used across the whole field of PCG-G from a large research body. Third, we map PCG-G methods to game content layers; it turns out that many of the methods used to generate game content from one layer can be used to generate content from another. We also survey the use of methods in practice, that is, in commercial or prototype games. Fourth and last, we discuss several directions for future research in PCG-G, which we believe deserve close attention in the near future.

\item[Web link:] \url{http://dl.acm.org/citation.cfm?id=2422957}
\item[Full text link:] \url{http://www.st.ewi.tudelft.nl/~iosup/pcg-g-survey11tomccap_rev_sub.pdf}
\item[Comments:] I found this from the ACM website, but i was unsure whether it would be an appropriate article because it is not specifically about a 2D platformer, but a more general article about Procedural Content Generation.
\end{description}

\bibliographystyle{ieeetr}
\bibliography{comp110_architecture}

\end{document}
